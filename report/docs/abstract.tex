\clearpage
\phantomsection
\addcontentsline{toc}{chapter}{Abstract}

%Must not exceed 600 words
\begin{abstract}
	
	\glsentryshort{edsac} was the first practical digital stored program computer, and ran its initial program in May 1949. It contributed much to the field of computing, and due to its notability is currently being reconstructed at \glsentrylong{tnmoc}.
	
	\glsentryshort{edsac} used delay line memory that stored data by inserting pulses into a delay medium and creating a feedback path from the output to the input. In the case of \glsentryshort{edsac} the delay medium consisted of very precisely machined steel tubes full of mercury. This is very difficult to reproduce authentically due constraints, both of budget and health and safety. The intention of this project is therefore to use modern technology to emulate this memory in a form and electrical interface indistinguishable from the original.
	
	The project has been successful in creating a digital emulation of a mercury delay line, implemented in hardware using a low cost \glsentryshort{fpga}. This digital delay line, coupled with analogue circuitry is able to interface with the high voltage signal used to drive the mercury delay lines. The delay line is also able to phantom power itself from the signal input with the addition of a single passive component to the main \glsentryshort{edsac} circuitry.
	
	In addition to the delay line itself, a comprehensive test harness has been designed that is able to emulate the signals produced by \glsentryshort{edsac}, and evaluate the performance of the delay line. This test harness interfaces to a \glsentryshort{pc} and displays the real-time status of the memory to a user.
	
	The delay line has been integrated in a physical enclosure matching that of the original, and has had its performance verified both using both the test harness produced by this project, as well as integration testing with the valve circuity of the main reconstruction effort.
	
	The project produces a well documented and tested methodology to implement delay line memory, that performs well enough to implement the memory of the reconstructed computer. Future work will formalise the construction of the delay line, and work to integrate it as part of the museum display.
	
	
	
\end{abstract}
