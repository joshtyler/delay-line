
\documentclass[12pt, a4paper, notitlepage]{report}

\usepackage{geometry}
\geometry{a4paper, portrait, margin=1in}

%Sort out generous chapter spacing
\usepackage{titlesec}
\titleformat{\chapter}[display]
{\normalfont\huge\bfseries}{\chaptertitlename\ \thechapter}{10pt}{\Huge} % Was:{\normalfont\huge\bfseries}{\chaptertitlename\ \thechapter}{20pt}{\Huge}
% Make {\scshape\Huge} for ALL chapters to have small caps headings
% Note this does not apply to ToC, but DOES apply to list of tables etc.
\titleformat{\section}
{\normalfont\Large\bfseries}{\thesection}{1em}{}
\titleformat{\subsection}
{\normalfont\large\bfseries}{\thesubsection}{1em}{}
\titleformat{\subsubsection}
{\normalfont\normalsize\bfseries}{\thesubsubsection}{1em}{}
\titleformat{\paragraph}[runin]
{\normalfont\normalsize\bfseries}{\theparagraph}{1em}{}
\titleformat{\subparagraph}[runin]
{\normalfont\normalsize\bfseries}{\thesubparagraph}{1em}{}

\titlespacing*{\chapter} {0pt}{-20pt}{10pt} %Was: {0pt}{50pt}{40pt}
\titlespacing*{\section} {0pt}{3.5ex plus 1ex minus .2ex}{2.3ex plus .2ex}
\titlespacing*{\subsection} {0pt}{3.25ex plus 1ex minus .2ex}{1.5ex plus .2ex}
\titlespacing*{\subsubsection}{0pt}{3.25ex plus 1ex minus .2ex}{1.5ex plus .2ex}
\titlespacing*{\paragraph} {0pt}{3.25ex plus 1ex minus .2ex}{1em}
\titlespacing*{\subparagraph} {\parindent}{3.25ex plus 1ex minus .2ex}{1em}
%\titlespacing{command}{left spacing}{before spacing}{after spacing}[right]
% spacing: how to read {12pt plus 4pt minus 2pt}
%           12pt is what we would like the spacing to be
%           plus 4pt means that TeX can stretch it by at most 4pt
%           minus 2pt means that TeX can shrink it by at most 2pt
%       This is one example of the concept of, 'glue', in TeX

%Command to apply to normal chapters to give them fancy small caps
%There will be a clever automatic way to do this, but I don't have all day...
\newcommand{\chapterStyle}[1]{\textsc{#1}}

%Remove chapter space in list of figures and tables
\usepackage{etoolbox}% http://ctan.org/pkg/etoolbox
\makeatletter
% \patchcmd{<cmd>}{<search>}{<replace>}{<succes>}{<failure>}
\patchcmd{\@chapter}{\addtocontents{lof}{\protect\addvspace{10\p@}}}{}{}{}% LoF
\patchcmd{\@chapter}{\addtocontents{lot}{\protect\addvspace{10\p@}}}{}{}{}% LoT
\makeatother


\def\singleQuotes#1{\lq{#1}\rq} %Macro to wrap something in single quotes

\usepackage{etoolbox} %Let's let bibliography be ragged right to avoid typesetting problems
\apptocmd{\thebibliography}{\raggedright}{}{}

\usepackage[table]{xcolor}

%Tables (and long tables)
\usepackage{longtable}

% alternate rowcolors for all long-tables

\let\oldlongtable\longtable

\let\endoldlongtable\endlongtable

\renewenvironment{longtable}{\rowcolors{2}{white}{gray!30}\oldlongtable} {

	\endoldlongtable}


%Drawing
\usepackage{tikz}

% http://tex.stackexchange.com/questions/177164/tikz-externalization-and-directory-questions
\usetikzlibrary{external}
\tikzexternalize[prefix=tikz/] 
% Only externalise on-demand.
\tikzexternaldisable

\usepackage{tikz-3dplot}
\usetikzlibrary{shapes.geometric, shapes.misc, arrows}
\usetikzlibrary{calc}
\usetikzlibrary{positioning}
\usetikzlibrary{patterns}
\usetikzlibrary{automata}

\usepackage[europeanresistors]{circuitikz}

% Setup some tikz defaults
\newcommand{\nodeDist}{1cm}
\newcommand{\blockWidth}{1.5*\nodeDist}
\newcommand{\blockHeight}{1.5*\nodeDist}
\newcommand{\tikzStateNodeDist}{3*\nodeDist}

\tikzstyle{block} = [rectangle, rounded corners, minimum width=\blockWidth, minimum height=\blockHeight, text centered, draw=black, fill=black!40]

\tikzset{font={\small}} %set small font

\tikzstyle{every state}=[fill=black!40,draw=none,minimum size=\tikzStateNodeDist/2] %Setup state machine style

\tikzstyle{stateMachine}=[->,auto,node distance=\tikzStateNodeDist,thick]

\tikzstyle{arrowNml} = [thick,->,>=stealth]
\tikzstyle{arrowRev} = [thick,<-,>=stealth]
\tikzstyle{arrowDbl} = [thick,<->,>=stealth]
\tikzstyle{lineNml} = [thick]


%Acronyms!
\usepackage[nonumberlist, acronyms, toc,]{glossaries}
\makenoidxglossaries
\setacronymstyle{long-short}
\renewcommand{\glsgroupskip}{} %Don't put a small space between groups
\newacronym{edsac}{EDSAC}{electronic delay storage automatic calculator}
\newacronym{eniac}{ENIAC}{electronic numerical integrator and computer}
\newacronym{tnmoc}{TNMoC}{The National Museum of Computing}
\newacronym{leo}{LEO}{Lyons electronic office}
\newacronym{fpga}{FPGA}{field programmable gate array}
\newacronym{dc}{DC}{direct current}
\newacronym{ac}{AC}{alternating current}
\newacronym{pcb}{PCB}{printed circuit board}
\newacronym{ram}{RAM}{random access memory}
\newacronym{hdl}{HDL}{hardware description language}
\newacronym{fifo}{FIFO}{first in, first out}
\newacronym{risc}{RISC}{reduced instruction set computing}
\newacronym{arm}{ARM}{advanced RISC machine} %This breaks if using \gls{risc}
\newacronym{uart}{UART}{univeral asynchronous receiver/transmitter}
\newacronym{gpio}{GPIO}{general purpose input/output}
\newacronym{ic}{IC}{integrated circuit}
\newacronym{rc}{RC}{resistor-capacitor}
\newacronym{cpld}{CPLD}{complex programmable logic device}
\newacronym{pll}{PLL}{phase locked loop}
\newacronym{lc}{LC}{logic cell}
\newacronym{lut}{LUT}{look-up table}
\newacronym{tqfp}{TQFP}{thin quad flat package}
\newacronym{vhdl}{VHDL}{VHSIC hardware definition language}
\newacronym{spice}{SPICE}{simulation program with integrated circuit emphasis}
\newacronym{lvcmos}{LVCMOS}{low voltage complementary metal oxide semiconductor}
\newacronym{gbp}{GBP}{gain-bandwidth product}
\newacronym{ldo}{LDO}{low-dropout}
\newacronym{rcbo}{RCBO}{residual-current circuit breaker with overcurrent protection}
\newacronym{rms}{RMS}{root mean square} %Import list of acronyms


%Degree symbol
\usepackage{textcomp} %Remove \gensymb warning
\usepackage{gensymb}

%Appendix Handling
\usepackage[titletoc,title]{appendix}

%Posh enumeration
\usepackage{enumitem}
\setlist[enumerate]{nosep} %No special separation for enumerated lists
\setlist[itemize]{nosep}

%Times for main font and math
\usepackage{mathptmx}

% AMS math packages:
% Required for proper math display.
\usepackage{amsmath,amsfonts,amsthm, amssymb}

% Graphicx
\usepackage[export]{adjustbox} %Also loads graphicsx
\graphicspath{ {figs/} }
\newcommand{\figureWidth}{0.7\textwidth}
\newcommand{\figureHeight}{0.7\textwidth}

\usepackage{xargs}
\newcommandx{\includegraphicssized}[2][2]{\includegraphics[max height=\figureHeight, max width=\figureWidth, #2]{#1}}

% Booktabs - better looking tables
\usepackage{booktabs}

% SI Units
\usepackage{siunitx}
\DeclareSIUnit \pixel{px}
\DeclareSIUnit \bit{b}


% Caption - better looking captions
\usepackage{caption}

% Hyperref:
% This package makes all references within your document clickable. By default, these references will become boxed and colored. This is turned back to normal with the \hypersetup command below.
\usepackage[pdfpagelabels]{hyperref} %pdfpagelabels allows page numbers such as i and ii in pdf reader
	\hypersetup{colorlinks=false,pdfborder=0 0 0}


%Timing diagrams

\usepackage{tikz-timing}[2011/01/09]

	
%Count pages
\usepackage{lastpage}

% headers
\usepackage{fancyhdr}
% we want fancy headers even when LaTeX explicitely uses "plain" pagestyle
\fancypagestyle{plain}{%
	% reset headers and footers
	\fancyhead{}
	\fancyfoot{}
	% top right corner of each page
	\fancyhead[R]{\scriptsize \thesisAuthor, \thesisDate, MSc Dissertation}
	\fancyfoot[C]{\scriptsize - \thepage\ -}
}
\pagestyle{plain} % Set all pages to use this style

\usepackage{dirtree} %View directory structures
\renewcommand*\DTstylecomment{\rmfamily\color{red}}

%\DTsetlength{offset}{width}{sep}{rule-width}{dot-size}
%Default is {0.2em}{1em}{0.2em}{0.4pt}{1.6pt} 
\DTsetlength{0.2em}{1em}{0.2em}{0.4pt}{3pt} %Increase dot weight

\usepackage{pdflscape} %Allow one page to be landscape
\usepackage{pgfgantt} %Gantt charts

\usepackage{multirow}
\usepackage{multicol}
\usepackage{hhline}

\usepackage{subcaption} %Subfigures

\usepackage{listings} %Listings of source code etc.

\usepackage{pdfpages} %Include pdfs verbatim

\usepackage{placeins} %FloatBarrier

%Stuff for reviewing, placeholders, etc.
%Commmand to review
\newcommand{\review}[1]
{
	{\color{red} \bfseries [#1]}
}
%Dummy figures
\usepackage{tcolorbox}
\newcommand{\dummyfigure}
{
	\begin{tcolorbox}[width=3.5in,height=2.5in,arc=0mm,auto outer arc, halign=center, valign=center]
		Figure placeholder
	\end{tcolorbox}
}
%Fake text
\usepackage{lipsum}
