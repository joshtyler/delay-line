\clearpage
\begin{appendices}
	
\chapter{Folder Structure} \label{sec:folder-struct}

The project is structured as a single repository using the git version control system. For those unfamiliar with git, the project folder may be treated as an ordinary computer folder, however there exists inside the root of the directory a hidden folder, \texttt{.git}, which contains the entire history of the project. This history is not of use to a user only interested in the latest state of the project, but is useful to find files or parts of files which may have been deleted. An example of such files are the project files for the Lattice and Xilinx toolchains, which were later replaced in the project by the open-source Project IceStorm toolchain.

The philosophy of the project repository is that it should not contain any:
\begin{itemize}
	\item Files not directly related to the project itself (e.g. reference material)
	\item Auxillary files automatically generated by tools
	\item Binary files that may be easily regenerated by tools
\end{itemize}
but should contain:
\begin{itemize}
	\item Design files
	\item Project files/configuration files used by tools
	\item Documentation files, even if they may be re-generated (e.g. \LaTeX generated \texttt{.pdf} files), provided they are not excessively bulky.
\end{itemize}
This philosophy is enforced inside the repository using \texttt{.gitignore} files.


The structure of each sub-folder in the project repository is shown Figure \ref{fig:repo-dir-struct} (in alphabetical order).

Note that unlabelled folders contain identical types of files to the same name of folder elsewhere in the hierarchy.

\begin{figure}[ht]
	\dirtree{%
		.1 admin \DTcomment{Administration files. E.g. purchase orders}.
		.1 demo \DTcomment{\LaTeX files used to generate the project demonstration slides}.
			.2 figs \DTcomment{Figures used in the slides}.
		.1 hdl \DTcomment{\Glsentryshort{hdl} design files/project files, and simulation testbenches}.
			.2 common \DTcomment{Common Verilog constructs used across multiple designs}.
				.3 sim \DTcomment{SystemVerilog Simulation testbenches}.
				.3 sim\_modelsim \DTcomment{ModelSim project file for simulation}.
				.3 src \DTcomment{Synthesizable Verilog code}.
			.2 delay\_line \DTcomment{Files related to the delay line design}.
				.3 icestorm \DTcomment{Makefile for the Project IceStorm toolchain}.
				.3 sim.
				.3 sim\_modelsim.
				.3 src.
			.2 test\_harness \DTcomment{Files related to the delay line design}.
				.3 icestorm.
				.3 sim.
				.3 sim\_modelsim.
				.3 src.
		.1 report \DTcomment{\LaTeX files used to generate the project report (this document)}.
			.2 docs \DTcomment{\LaTeX source files that are not the master project file}.
			.2 figs.
			.2 refs \DTcomment{Files for \textsc{Bib}\TeX referencing}. 
		.1 software \DTcomment{Supporting software intended to run on a \glsentryshort{pc}}.
			.2 mem\_gui \DTcomment{Source related to the \glsentryshort{gui} software that talks to the test harness}.
				.3 {.idea} \DTcomment{Project files for CLion}.
				.3 ftdi\_wrapper \DTcomment{Source for wrapping the FTDI driver}.
				.3 generic\_classes \DTcomment{Source for classes used in various parts of the project}.
				.3 main\_gui \DTcomment{Source to implement \glsentryshort{gui} itself}.
				.3 memory\_manager \DTcomment{Source to monitor each unit of \glsentryshort{edsac} memory}.
				.3 status\_manager \DTcomment{Source to control the state of the system}.
				.3 uart\_manager \DTcomment{Source to control the transmission and receipt of messages}.
				.3 uart\_msg \DTcomment{Source relating to \glsentryshort{uart} message formats}.
			.2 uart\_msg\_const\_gen \DTcomment{A script to generate \glsentryshort{uart} message constant header files}.
				.3 cpp\_out \DTcomment{Output folder for C++ header(s)}.
				.3 verilog\_out \DTcomment{Output folder for Verilog header(s)}.
		.1 spice \DTcomment{LTspice simulation schematics}.
			.2 component\_models \DTcomment{Models for various components used in the design}.
	}
	\caption{Repository Directory Structure}
	\label{fig:repo-dir-struct}
\end{figure}


\chapter{\glsentryshort{hdl} Code Overview}

\review{Is this section really necessary?}
	
	
\chapter{Schematics}

\end{appendices}